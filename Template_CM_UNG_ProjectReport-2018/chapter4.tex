
\chapter{CONCLUSIONS}

In this EDA project, we aimed to gain insights and streamline the dataset by applying various techniques including statistical analysis, graphical visualization, and domain knowledge. Our objective was to reduce the dimensionality of the dataset while retaining important and informative features. Through a comprehensive analysis, we successfully reduced the dataset from 45 columns to 16 columns.
\begin{table}[htpb]
	\caption{Remaining Columns}
	\centering
	\begin{tabular}{|l|}
		\hline
		Patient Age\\
		Genes in mother's side\\
		Inherited from father\\
		Maternal gene\\
		Paternal gene\\
		Blood cell count mcL\\
		Mother's age\\
		Father's age\\
		Number of previous abortion\\
		White Blood cell count thousand per microliter\\
		Symptom 1 \\
		Symptom 2\\
		Symptom 3\\
		Symptom 4\\
		Symptom 5 \\                          
		Disorder Subclass\\ 
		\hline
	\end{tabular}
	\label{tab 2}
\end{table}
\newline
\noindent
Statistical analysis played a crucial role in identifying key features that significantly contributed to the dataset. By evaluating statistical measures such as mean, standard deviation, and correlation coefficients, we were able to quantify the relationships between variables and make informed decisions on feature selection.\\
\noindent
Graphical visualization techniques were employed to visualize patterns, trends, and outliers in the data. Box plots, scatter plots, and histograms provided valuable insights into the distribution and relationships of variables. By visually examining the data, we gained a deeper understanding of its characteristics, which guided us in selecting the most relevant features.\\
\noindent
Domain knowledge also played a significant role in feature selection. By leveraging our understanding of the subject matter, we were able to identify domain-specific attributes that were essential for analysis and decision-making. This domain-based information added an additional layer of context and relevance to the feature selection process.\\
\noindent
Our approach was further supported by citable sources, including research papers that discuss the importance of feature selection, dimensionality reduction, and the impact of domain knowledge in data analysis. These sources provided a theoretical foundation and best practices for our EDA project.\\
\noindent
Additionally, we implemented a dashboard to present the analyzed data in a visually appealing and interactive manner. The dashboard allowed users to explore the reduced dataset, visualize trends, and gain actionable insights efficiently. By incorporating user-friendly features and interactive elements, the dashboard facilitated data-driven decision-making and enhanced the overall user experience.\\
\noindent
In conclusion, our EDA project successfully reduced the dataset from 45 columns to 16 columns through the application of statistical analysis, graphical visualization, and domain-based information. The feature selection process was supported by citable sources, ensuring the validity and reliability of our approach. The implementation of a dashboard provided an effective means of visualizing the analyzed data and empowered users to make informed decisions based on the insights gained.\\ 
\noindent
Future work could involve further refinement of the feature selection process, exploration of advanced visualization techniques, and integration of machine learning algorithms for predictive modeling based on the reduced dataset.




